\documentclass{beamer}

% Used packages
\usepackage{graphicx}
\usepackage{hyperref}
\usepackage{algorithm}
\usepackage{algpseudocode}

% The title
\title[Code Mobility]{Code Mobility}

% The date
\date{\today}

% The author
\author[Jakovljevi\'c,Selyunin,Pelesi\'c]{
 \Large{Konstantin Selyunin}\\
  \small{\texttt{e1228206@student.tuwien.ac.at}}\\
 \Large{Miljenko Jakovljevi\'c}\\
  \small{\texttt{e1228206@student.tuwien.ac.at}}\\
  \Large{Igor Pelesi\'c}\\
  \small{\texttt{e0006828@student.tuwien.ac.at}}\\
}

% Use Warsaw theme
\usetheme{Warsaw}

% New commands
\newcommand{\mc}[1]{$\mathcal{#1}$}

\theoremstyle{definition} \newtheorem{mdefinition}{Definition}
\theoremstyle{plain} \newtheorem{mtheorem}{Theorem}
\theoremstyle{plain} \newtheorem{mcorollary}{Corollary}
\theoremstyle{plain} \newtheorem{mfact}{Fact}

% Begin of document
\begin{document}

% \begin{frame}
% 	\frametitle{}
% % 	\framesubtitle{}
% 	\begin{block}{}
% 		\begin{itemize}
% 			\item	
% 		\end{itemize}
% 	\end{block}
% \end{frame}

\begin{frame}
	\titlepage
\end{frame}

\begin{frame}
	\frametitle{Outline}
	\tableofcontents
\end{frame}

\section{Preliminaries}

\begin{frame}
	\frametitle{Resolution as a Total Function}
% 	\framesubtitle{}
	\begin{block}{Input parameters}
		\begin{description}
			\item[$q$]	a literal, called the {\it clashing literal}
			\item[$C$]	an arbitrary clause
			\item[$D$]	an arbitrary clause
		\end{description}
	\end{block}
	
	\begin{block}{Result}
		\begin{description}
			\item[$R$]	the resulting clause, called the {\it resolvent}
		\end{description}
	\end{block}
\end{frame}

\begin{frame}
	\frametitle{Resolution as a Total Function (ctd.)}
% 	\framesubtitle{}
	\begin{block}{Definition of {\bf res}}
		If $C = \{q\} \cup \alpha$ and $D = \{\neg q\} \cup \beta$ are two regular clauses, then
		\begin{equation*}
			{\bf res}(q,C,D) =	\begin{cases}
														\alpha \cup \beta	& \text{if } \alpha \cup \beta \text{ is consistent,}\\
														\top							& \text{otherwise.}
													\end{cases}
		\end{equation*}
		
		$\top$, the (unique) tautologous clause, is used as an identity element of {\bf res}:
		\begin{equation*}
			{\bf res}(q,C,\top) =	C.
		\end{equation*}
	\end{block}
\end{frame}

\begin{frame}
	\frametitle{Resolution as a Total Function (ctd.)}
% 	\framesubtitle{}
	\begin{block}{Definition of {\bf res} (ctd.)}
		If $C$ does neither contain $q$ nor $\neg q$ and $D = \{q\} \cup \beta$ or $D = \{\neg q\} \cup \beta$, then
		\begin{equation*}
			{\bf res}(q,C,D) = C.
		\end{equation*}
		
		If neither $C$ nor $D$ contains $q$ or $\neg q$, then
		\begin{equation*}
			{\bf res}(q,C,D) = 	\begin{cases}
														C		& \text{if } C < D,\\
														D		& \text{otherwise.}
													\end{cases}
		\end{equation*}
	\end{block}
\end{frame}









%\begin{frame}
%	\frametitle{Example (ctd.)}
%	\framesubtitle{Initial resolution DAG G}
%	\includegraphics[width=\textwidth]{./pics/res-dag-init.pdf}
%\end{frame}

\
\section{References}

\begin{frame}
	\frametitle{References}
% 	\framesubtitle{}

	\begin{itemize}
		\item	A. Van Gelder: Pool Resolution and Its Relation to Regular Resolution and DPLL with Clause Learning. In: LPAR 2005, LNAI 3835, pp. 580-594
		\item  Krishnamurthy, B.: Short proofs for tricky formulas. Acta Informatica 22 (1985) 253-274
		\item St\"almark, G.: Short resolution proofs for a sequence of tricky formulas. Acta Informatica 33 (1996) 277-280
		\item Alekhnovich, M., Johannsen, J., Pitassi, T., Urquhart, A.: An exponential separation between regular and unrestricted resolution. In: Proc. 34th ACM Symposium on Theory of Computing. (2002) 448-456
		\item Beame, P., Kautz, H., Sabharwal, A.: Towards understanding and harnessing the potential of clause learning. Jorunal of Atrificial Intelligence Research 22 (2004) 319-351
	\end{itemize}
\end{frame}

% End of document
\end{document}
