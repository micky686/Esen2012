\documentclass[12pt,a4paper,titlepage,oneside]{article}
\usepackage{ese_prot}

\title{Beispiel 1}

\author{ Vorname Nachname, \matrnr 6666666    \\
         {\small e6666666@student.tuwien.ac.at}
}

\begin{document}

% create titlepage
\maketitle

\tableofcontents
\newpage

\section{Beispiel 1.1A}

\subsection{Aufgabenstellung}

Kopieren Sie die originale Aufgabenstellung in das Protokoll.

\subsection{Kommentare zur Aufgabenstellung (optional)}

Begr�ndung f�r etwaige Abweichungen (muss mit Tutor abgesprochen
sein)

\subsection{Implementierung}

Die Implementierung besteht aus folgenden Teilen ...

Das Makro zum Auslesen der Speicherstellen wurde wie folgt
implementiert:

\begin{lstlisting}{}
    /* Port B auf Output und einschalten */
/*==========================================================================*/
/*23456789012345678901234567890123456789012345678901234567890123456789012345*/
/* lange Zeilen so wie der Text dieser Zeile der eigentlich f�r einen Kommentar viel zu lang ist, werden umgebrochen */
/*==========================================================================*/

   outp(0xFF,PORTB);
   outp(0xFF,DDRB);
\end{lstlisting}

\subsection{Ergebnisse (optional)}

Falls Messergebnisse (Kurven, etc.) in der Aufgabenstellung
gefordert sind.

Abbildung~\ref{figPlot} zeigt die Kennlinie ...

\begin{figure}[htb]
\epsfxsize=0.8\textwidth \hfill \epsfbox{figures/plot} \hfill\hbox{}
\caption{Kennlinie ...} \label{figPlot}
\end{figure}


\subsection{Probleme und Fallstricke}

Alle Probleme, die Sie hatten. Dieses Kapitel schon von Anfang an
mitziehen, damit nicht Probleme nach der L�sung in Vergessenheit
geraten. Zu gel�sten Problemen immer die L�sung schreiben.

\subsection{Arbeitsaufwand}

Tabelle mit geplantem Aufwand (aus Designprotokoll) und
tats�chlichem Aufwand aufgeschl�sselt nach Treiberfunktionen,
Integration und Gesamtaufwand

\subsubsection{Speicherresourcenanalyse}

Die Ermittlung des statischen Speicherverbrauchs erfolgt �ber das
.elf-File.

Die Ermittlung des dynamischen Speicherverbrauchs erfolgt durch das
Speicherresourcenanalyseprogramm auf der Homepage.

\begin{tabular}{|l|l|ll|}
\hline
  Typ & Insgesamt & Verwendet & \\ \hline \hline

  Flash ROM & 8192 Byte & 2315 Byte &(28\%) \\ \hline
  SRAM & 256 Byte & statisch 6 Byte &(2,3\%) \\
       &          & dynamisch 24 Byte &(9,4\%) \\
       &          & insgesamt 30 Byte &(11,7\%) \\ \hline
  EEPROM & 256 Byte & 0 Byte &(0\%) \\ \hline
\end{tabular}

\section{Listings}

der von Ihnen implementierte Code, anbei als Beispiel die mem\_eval
Listings.

\subsection{mem\_eval.h}

\lstinputlisting{testfiles/mem_eval.h}

\newpage

\subsection{mem\_eval.c}

\lstinputlisting{testfiles/mem_eval.c}


\end{document}
